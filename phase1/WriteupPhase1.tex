
\def\pathToRoot{../}\input{\pathToRoot headers/uebungHeader}

\usepackage{charter}
\fontfamily{bch}\selectfont

\begin{document}
\title{What is a Category?}
\author{Sarah Mameche, Leonhard Staut, Andreas Meyer}
\maketitle

\section {Definition of Categories}
\begin{definition}{Category}
	A \emph{category} $\mathscr{C}$ has the following components:
	\begin{itemize}
		\item a collection of \emph{objects}, denoted by $ob(\mathscr{C})$
		\item $\forall \; A, B \in ob(\mathscr{C})$, a collection of \emph{maps} (\emph{arrows}, \emph{morphisms}), denoted by $\mathscr{C}(A, B)$
		\item $\forall \; A, B, C \in ob(\mathscr{C})$, a \emph{composition}
		\[\mathscr{C}(B, C) \times \mathscr{C}(A, B) \rightarrow  \mathscr{C}(A, C)
		\]		\[ (g,f) \mapsto (g \of f)
		\]
		\item $\forall A \in ob(\mathscr{C})$, an element $1_A \in \mathscr{C}(A, A)$, called the \emph{identity} on $A$ 
 	\end{itemize}
 The following axioms have to be satisfied:
 \begin{itemize}
 	\item identity laws: $\forall f \in \mathscr{C}(A, B). \; f \of 1_A = f = 1_B \of f$ 
 	\item associativity: $\forall f \in \mathscr{C}(A, B), \; g \in \mathscr{C}(B, C), \; h \in \mathscr{C}(C, D). \; (h \of g) \of f = h \of (g \of f) $.
 \end{itemize}
\end{definition}

%%We also write $A \in \mathscr C$ instead of $A \in ob(\mathscr{C})$, and we write
%%$A \overset{f}{\rightarrow} B$ for $f \in \mathscr{C}(A,B)$

\section {Diagrams}
It is often possible and useful to draw diagrams for a category or parts thereof.
For example consider the following diagram.
\[
  \begin{tikzcd}
    A \arrow{r}{f} \arrow{d}{g} & B \arrow{d}{h} \\
    C \arrow{r}{i}              & D 
  \end{tikzcd}
\]
This represents a category with objects $\{A, B, C, D\}$ and
morphisms $\{f, g, h, i, 1_A, 1_B, 1_C, 1_D \}$.
Since every object is required to have an identity arrow by definition,
we usually leave them implicit and do not include them in diagrams.\\
We say a diagram \emph{commutes} if any two paths between two objects
obtained by composing arrows are identical.
The example diagram above commutes if $ h \circ f = i \circ g $, meaning there is
exactly one way to reach $D$ from $A$.

\section {Isomorphisms}
\begin{definition}{Isomorphism}
  Let $\mathscr{C}$ be a category and $A, B \in ob(\mathscr{C})$ objects. \\
  A map $f \from A \to B \in \mathscr{C}$ is called an \emph{isomorphism} if there is a $g \from B \to A \in \mathscr{C}$,
  s.t. $g \circ f = 1_A$ and $f \circ g = 1_B$. \\
  $A$ and $B$ are \emph{isomorphic} in $\mathscr{C}$ ($A \cong B$) if there is an isomorphism between them.
\end{definition}

In short, a morphism is an isomorphism if it has an inverse.
If a map $f$ is an isomorphism, then its inverse is unique. Thus we speak of
\emph{the} inverse of $f$.

\begin{proof}
  Let $f \from A \to B$ be a map.
  Assume there exist two inverses $g \from B \to A$, such that $gf = 1_A$, $fg = 1_B$ and \mbox{$h \from B \to A$}, such that $hf = 1_A$, $fh = 1_B$.
  We know from the identity axiom: $1_A \of g = g$ and $h \of 1_B = h$. Therefore we get:
 \[  1_A \of g = g \textiff (h \of f) \of g = g \textiff h \of (f \of g) = g \textiff h \of 1_B = g \textiff h=g. \]
\end{proof}


\section {Examples of Categories}
We now introduce some constructions of categories as an example to show
how categories are used to describe basic mathematical structures.
\begin {enumerate}
 \item The category of sets, denoted as \textbf{Set}, is the category whose objects $ob$(\textbf{Set}) are sets.
   \\ The arrows in \textbf{Set} are the functions between two $A,B\ \in ob(\textbf{Set}).$
   \\ The identity- function is defined as $\forall A: set, \ id_A:\ A \to A,\ \forall x \in A \ f \ x = x$
   \\ The composition $\circ$ is the composition of functions and this is associative. Proof:
   \\ Assume sets $A, B, C, and D$ and functions $f \from A \to B$, $g \from B \to C$, $h \from C \to D$ and $x \in A$.
   \\ $((h\ \circ \ g)\ \circ\ f)\ x = ((h \ \circ g \ )(f \ x) =h(g(f\  x)) $
   \\ $(h\ \circ \ (g \ \circ f))\ x = h ((g\ \circ \ f) \ x) = h(g(f \ x))$
   \\ Thus $(h\ \circ \ g)\ \circ \ f = \ h \ \circ \ (g \ \circ \ f)$ holds. $\Box$
 \item The category of relations, denoted as \textbf{Rel}, is the category whose objects are the sets
   \\The arrows are all binary relations between two $A,B \ \in ob(\textbf{Rel})$
   \\The identity arrow  is the identity function $\forall A: set, \ id_A:\ A \to A,\ \forall x \in A \ f \ x = x$
   \\ The composition $R \circ S$, $R  \in \textbf{Rel}(A,B), S \in \textbf{Rel}(B,C), A,B,C \in ob(\textbf{Rel})$
   is defined as $ (x,y) \in  R \circ S \leftrightarrow \exists  z.(x,z) \in S \land (z,y) \in R$
   
 \item Categories similar to \textbf{Set} can be constructed for sets which have some additional structure and structure-preserving mappings between them.
   For example, there is a category \textbf{Poset} with partially ordered sets as objects. It is defined similarly to \textbf{Set}, with the difference that the functions are monotone. 
  
 \item A single poset $(P, \leq)$ also gives rise to a category $\mathscr{P}$ if we take elements of $P$ to be the objects.
   There is a unique arrow in $\mathscr{P}(A, B)$ iff $A \leq B$.
   The reflexivity requirement of $\leq$ ensures that identity arrows exist for all objects. Also, the category has composition because the order is transitive.
  
  \item \textbf{Vect$_k$} is the category of vector spaces over a field $k$, with linear transformations as arrows.
  
  \item There is a category \textbf{Grp} whose objects are groups; maps are given by group homomorphisms between two groups $A,B \in ob(\textbf{Grp}).$
    Similarly there are categories \textbf{Mon} with monoids as objects and \textbf{Ring} with rings as objects.
    The maps are given by homomorphisms between monoids and rings respectively.
  
  \item
    The objects in a category do not have to be like sets, and the maps do not have to be like functions.
    A single group $(G, \cdot)$ can also be seen as a category $\mathcal{G}$. The category has only one object $G$ which represents the group itself,
    and arrows in $\mathcal{G}(G, G)$  correspond to group elements.
    The identity arrow is the unit element $1$ of the group.\\
    Composition of maps in the category $\circ$ corresponds to applying the group action $\cdot$.\\
    A group requires all elements to have an inverse, thus we need each arrow $\mathcal{G}(G, G)$ to be an isomorphism.\\
    If we dropped the last requirement, the category would represent a single monoid.
  
  \item We often use categories to represent mathematical objects and connections between them.
    However, categories do not need to represent mathematical structures at all.
    They can be arbitrarily constructed by giving objects,
    arrows, and composition in a way that satisfies the axioms in the definition.
    It is not necessary to give those objects, arrows, or composition any meaning.
    For instance, the category \textbf{1} contains a single object and its identity map,
    without further specification what the object is.\\
    A category that has no arrows besides identities is called \emph{discrete}.\\
      \[
        \begin{tikzcd}
          \bullet \arrow[loop right, "1"]\\
        \end{tikzcd}
      \]
      It is easy coming up with more examples of categories which do not have any mathematical content:\\[3mm]
      \begin{minipage}{.5\linewidth}
        \[
          \begin{tikzcd}
            \bullet \arrow[loop left, "1"] \qquad \bullet \arrow[loop right, "1"]\\
          \end{tikzcd}
        \]
      \end{minipage}%
      \begin{minipage}{.5\linewidth}
        \[
          \begin{tikzcd}
            A \arrow{r}{g} \arrow[swap, dr, "f \circ g"] & B \arrow{d}{f} \\
            & C
          \end{tikzcd}
        \]
      \end{minipage}
    \end {enumerate}

    \newpage

\section {Monics and epics}

Monomorphisms and epimorphisms are two special kinds of morphisms.

\begin{definition}{Monomorphism}
  Let $\mathscr{C}$ be a category. A map $X \overset{f}{\to} Y$ is called a \emph{monomorphism} if for all objects $Z$ and maps
  $
  \begin{tikzcd}
    Z \arrow[r,shift left, "g"] \arrow[r,shift right,swap,"g'"] & X
  \end{tikzcd}
  $
  \[
    f \circ g = f \circ g' \Rightarrow g = g'
  \]
  If $f$ is a monomorphism, we say $f$ is \emph{monic}.
\end{definition}

\begin{definition}{Epimorphism}
  Let $\mathscr{C}$ be a category. A map $X \overset{f}{\to} Y$ is called an \emph{epimorphism} if for all objects $Z$ and maps
  $
  \begin{tikzcd}
    Y \arrow[r,shift left, "g"] \arrow[r,shift right,swap,"g'"] & Z
  \end{tikzcd}
  $
  \[
    g' \circ f = g' \circ f \Rightarrow g = g'
  \]
  If $f$ is an epimorphism, we say $f$ is \emph{epic}.
\end{definition}

%% TODO: We could add a reference to the definition of categories, so that we can link it in the text below

In other words, a monomorphism is cancellable on the left, and an epimorphism is cancellable on the right of compositions.
By the identity laws, for any object $A$ the identity map $1_A$ is always both monic and epic.\\
In the category \textbf{Set}, monic maps are exactly the injective functions, and epic maps are exactly the surjective functions.
\begin{proof}
  First we show $f$ monic $\Leftrightarrow$ $f$ injective.
  \begin{itemize}
  \item $"\Rightarrow"$: Let $f \from A \to B$ be monic, $a, a' \in A$ with $a \neq a'$, $g, g' \from {x} \to A$ with $g(x) = a, g'(x) = a'$.
    We show $a \neq a' \Rightarrow f(a) \neq f(a')$.
    We have $g \neq g'$.
    Since $f$ is monic we get $f \circ g \neq f \circ g'$.\\
    We have: $f(a) = f(g(x)) \neq f(g'(x)) = f(a')$. Therefore $f$ is injective.
  \item $"\Leftarrow"$: Let $f \from A \to B$ be injective, $g, g' \from X \to A$ with $g \neq g'$.
    We show $f \circ g \neq f \circ g'$.
    There exists an $x \in X$ with $g(x) \neq g'(x)$.\\
    We have: $f(g(x)) \neq f(g'(x))$ since $f$ is injective.
    Therefore $f$ is monic.
  \end{itemize}
  Secondly we show $f$ epic $\Leftrightarrow$ $f$ surjective.
  \begin{itemize}
  \item $"\Rightarrow"$: Let $f \from A \to B$ be epic, and $X$ a two element set, e.g. $\{ \texttt{true}, \texttt{false}\}$.
    Let $g \from B \to X$ be the characteristic function of \texttt{Im}$(f)$ which is defined by
    $\forall b \in B. \ g(y) = \texttt{true}
    \Leftrightarrow y \in \texttt{Im}(f)$.
    And let $g' \from B \to X$ be the constant \texttt{true}-function.
    $\forall b \in B. \ g'(y) = \texttt{true}$.
    Note that $g = g'$ exactly if $f$ is surjective.\\
    We have: $g \of f = g' \of f \xRightarrow{f \ epic} g = g'$.
    Therefore $f$ is surjective.
  \item $"\Leftarrow"$: Let $f \from A \to B$ be surjective, and $g, g' \from B \to X$ with $g \neq g'$.
    There exists a $b \in B$ with $g(x) \neq g'(x)$.
    Since $f$ is surjective, there is an $a \in A$ with $f(a) = x$.\\
    We have: $g(f(a) \neq g'(f(a)) \Rightarrow g \circ f \neq g' \circ f$.
    Therefore $f$ is epic.
  \end{itemize}
\end{proof}
In ordinary set theory these properties of functions (injectivity, surjectivity) are formulated in terms of elements of a set.
Epic and monic maps enable us to abstract from that and
talk about these properties without referring to sets or their elements at all.
We can then use this abstract concept of monic and epic arrows to find these structures
in different categories.
In the categories \textbf{Mon}, \textbf{Grp}, \textbf{Ring}, and \textbf{Vect}
monic maps are exactly the injective homomorphisms,
similar to the category \textbf{Set}.
However, we show that there are already maps in \textbf{Mon} that are epic, but not surjective.
\begin{proof}
  Consider the two monoids $(\mathbb{N}, +, 0)$ and $(\mathbb{Z}, +, 0)$, i.e. the
  additive monoids of the natural numbers and the integers respectively.
  The map $i \from \mathbb{N} \to \mathbb{Z}; \ i(n) = n$ is called \emph{inclusion map},
  since the monoid $(\mathbb{N}, +, 0)$ is simply embedded in $(\mathbb{Z}, +, 0)$.
  Obviously this map is a monoid homomorphisms that is not surjective,
  because there are no negative integers in the image of $i$.
  But we show that $i$ is epic.
  Let $(\mathcal{M}, *, e)$ be any other monoid and $f, g \from \mathbb{Z} \to \mathcal{M}$ two
  monoid homomorphisms.
  We need to show the implication $f \circ i = g \circ i \Rightarrow f = g$, that is,
  if $f$ and $g$ agree on $\mathbb{N}$, then they agree on the entire domain $\mathbb{Z}$.\\
  Note that:
  \begin{align*}
    f(-n) &= f ((-1)_1 + (-1)_2 + \dots + (-1)_n)\\
          &= f(-1)_1 * f(-1)_2 * \dots * f(-1)_n
  \end{align*}
  Therefore it suffices to show that $f(-1) = g(-1)$.
  \begin{align*}
    f(-1) &= f(-1) * e \\
          &= f(-1) * g(0) \\
          &= f(-1) * g(1 - 1) \\
          &= f(-1) * g(1) * g(-1) \\
          &\overset{(*)}{=} f(-1) * f(1) * g(-1) \\
          &= f(-1 + 1) * g(-1) \\
          &= f(0) * g(-1) \\
          &= e * g(-1) \\
          &= g(-1) \\ 
  \end{align*}
  In $(*)$ we use our assumption that $f=g$ if restricted to $\mathbb{N}$.\\
  Therefore $i$ is an epic map, even though it is not surjective.
\end{proof}

Now we study how monomorphisms and epimorphisms are related to isomorphisms.
In the category \textbf{Set} isomorphisms are exactly the invertible functions.
It can be shown that invertible functions between two sets are
exactly the bijective functions, i.e. functions that are both injective and surjective.
Therefore, isomorphisms in the category \textbf{Set} are exactly those maps that are both monic and epic.\\
In the category \textbf{Grp} isomorphisms are exactly the bijective group homomorphisms.
since those are the group homomorphisms that have an inverse homomorphism.\\
Similarly, in the category \textbf{Ring} isomorphisms are exactly the bijective ring homomorphisms.\\
We now show for a general category $\mathscr{C}$ that every isomorphism is both monic and epic.
\begin{proof}
  Consider the commuting diagram:
  \[
    \begin{tikzcd}
      A \arrow[r,shift left, "y"] \arrow[r,shift right,swap,"x"] &
      B \arrow[dr, shift left, "1"]  \arrow{r}{m}  &
      C \arrow{d}{e} \arrow[r,shift left, "i"] \arrow[r,shift right,swap,"j"] & D\\
      &  & B \arrow[ul, shift left, "1"]&
    \end{tikzcd}
  \]
  $m \from B \to C$ is an isomorphism with inverse $e \from C \to B$.
  Then:\\
  \begin{minipage}{.5\linewidth}
    \vspace{4mm}
    \centering $m$ is monic:
    \[
      \begin{aligned}
        m \of x &= m \of y\\
        e \of (m \of x) &= e \of (m \of y)\\
        (e \of m) \of x &= (e \of m) \of y\\
        1_B \of x &= 1_B \of y\\
        x &= y
      \end{aligned}
    \]
  \end{minipage}%
  \begin{minipage}{.5\linewidth}
    \vspace{4mm}
    \centering $m$ is epic:
    \[
      \begin{aligned}
        i \of m &= j \of m \\
        (i \of m) \of e &= (j \of m) \of e\\
        i \of (m \of e) &= j \of (m \of e)\\
        i \of 1_C &= j \of 1_C\\
        i &= j
      \end{aligned}
    \]
  \end{minipage}%
\end{proof}
Note that the converse of the statement holds for the category \textbf{Set}, but not in general.
If a map is both monic and epic it is not necessarily an isomorphism.
For a map $f$ to be an isomorphism there needs to exist an inverse map $f^{-1}$.
However, monic and epic maps do not necessarily have an inverse.\\

\section {Dual categories}

$\mathscr{C}^{op}$ denotes the dual category for any category $\mathscr{C}$.
which is obtained by reversing the arrows in $\mathscr{C}$.
For every theorem $\Sigma$ in the language of category theory,
the dualized theorem $\Sigma^*$ exists,
and a proof for any theorem yields a rather simple proof for the dual theorem by the duality principle.\\
Dual categories are especially useful for structures that are inherently dual.
As an example, epic and monic maps are dual to each other, since any monic map in $\mathscr{C}$
is an epic map in $\mathscr{C}^{op}$ and vice versa.
The next theorem is an example where duality can be used.
We first show that the composition of two monic maps is monic.
\begin{proof}
  Consider the commuting diagram:
  \[
    \begin{tikzcd}
      A \arrow[dr, swap, "h"] \arrow{r}{g} & B \arrow{d}{f} \\
      & C
    \end{tikzcd}
  \]
  We assume $f$ and $g$ to be monic and show that $h = f \circ g $ is monic.
  Let $X$ be any object and $x, x' \from X \to A$ two maps in the category.
  \begin{align*}
    h \circ x &= h \circ x' \\
    (f \circ g) \circ x &= (f \circ g) \circ x' \\
    f \circ (g \circ x) &= f \circ (g \circ x') \\
    g \circ x &= g \circ x' \\
    x &= x' \\
  \end{align*}
  Therefore $h$ is monic.
\end{proof}
The dual of the statement above is that the composition of epic maps is also epic.
We get this fact immediately using the duality principle, since epic maps in $\mathscr{C}$
are monic maps in $\mathscr{C}^{op}$ whose composition is monic in $\mathscr{C}^{op}$, as we
have just shown, and therefore epic in $\mathscr{C}$.

\section {Terminal and Initial Objects}
\begin {definition}{Initial and terminal Objects}
  Let $\mathscr{C}$ be a category.\\
  An object $0$ is called \emph{initial}  iff for any object $A \in \mathscr{C}$,
  there is a unique morphism $0 \to A$.\\
  An object $1$ is called \emph{terminal} iff for any object $A \in \mathscr{C}$,
  there is a unique morphism $A \to 1$.
\end {definition}

Note that inital and terminal objects are dual to each other.
An initial object in $\mathscr{C}$ is terminal in $\mathscr{C}^{op}$.
A category in which the terminal is identical to the initial object are called a \emph{pointed category}.
Such an object is called \emph{zero object}.\\
In a general category, initial and terminal objects do not need to exist at all,
but if they do, then they are unique up to isomorphism.

\begin{proof}
Assume $0$ and $0'$ are both inital objects in some category $\mathscr{C}$ and show that
$f \from 0 \to 0', g \from 0' \to 0$ form an isomorphism
$f \circ g$ between $0$ and $0'$. Consider the following diagram:

\[
\begin{tikzcd}
  & 0' \arrow{dr}{g} \arrow{rr}{id_{0'}}    &&  0'\\
  0 \arrow{ur}{f} \arrow{rr}{id_0} && 0 \arrow{ur}{f}
\end{tikzcd}
\]

Since $0$ is initial, we know that $f$ is unique, from the same argument follows uniqueness of $g = f^{-1}$.
Therefore $f \circ g$ and $g \circ f$ is unique.\\
The same holds for terminal objects by duality.
\end{proof}


\subsection {Examples:}
In \textbf{Set} $\emptyset$ is initial and the one- element set $\{x\}$ terminal.
\begin{proof}
  \
\begin {itemize}
\item There is only the empty function from $\emptyset$ to any other set, since there are no arguments in the domain.
\item For all sets $A$ there is only one possible function $f \from A \to \{x\}$, since all elements of $A$ can only be mapped
  to $x$, that is, $\forall y \in A,\ f \ y = x$.
\end{itemize}
\end{proof}
In \textbf{Grp} the trivial group is both initial and terminal.
\begin{proof}
Let $1=\{e\}$ the trivial group with operation $*: e*e := e$.
\begin {itemize}
\item Let $\left({G, \circ}\right)$ be any group with identity element $e_G$,
  and let $\phi: 1 \to G$ be a function.
  Since group homomorphisms have to preserve the identity,
  there is only one possible mapping for $\phi$ to be a group homomorphism:
  $\phi (e) = e_G$.\\
  Now we only need show that $\phi$ is a group homomorphism.
  \begin{align*}
    \phi (e) \circ \phi (e) &= e_G \circ e_G \\
                            &= e_G \\
                            &= \phi (e) \\
                            &= \phi (e * e)
  \end{align*}
  Therefore $\phi$ is the unique group homomorphism from the trivial group.
\item Recall that the singleton set is terminal in \textbf{Set}.
  Therefore there is exactly one mapping
  $\phi: G \to \{e\}$ defined by:
  $\forall g \in G: \phi(g) = e$.
  Again we need show that $\phi$ is a group homomorphism.
  For any $g, g' \in G$, we have:
  \begin{align*}
    \phi(g)*\phi(g') &= e * e \\
    &= e \\
    &= \phi (g \circ g')
  \end{align*}
  Therefore $\phi$ is the unique group homomorphism into the trivial group.
\end{itemize}
The trivial group is both initial and terminal. It is therefore an example for a
zero object, and \textbf{Grp} is an example for a pointed category.
\end{proof}
The idea behind the name pointed category is that the shape of \textbf{Grp}, if you were to draw it,
looks like all other objects are pointed at the trivial group.
\[
  \begin{tikzcd}[column sep=large]
    \dots
    & \bigcirc \arrow [drrr, ""]
    & \bigcirc \arrow[drr, ""]
    & \bigcirc \arrow[dr, ""]
    & \dots
    & \bigcirc \arrow[dl, ""]
    & \bigcirc \arrow[dll,""]
    & \bigcirc \arrow[dlll,""]
    & \dots \\[5mm]
    \dots
    & \bigcirc \arrow[rrr, ""]
    & & &
    Triv
    \arrow[dlll, ""]
    \arrow[dll, ""]
    \arrow[dl, ""]
    \arrow[dr, ""]
    \arrow[drr, ""]
    \arrow[drrr, ""]
    \arrow[ulll, ""]
    \arrow[ull, ""]
    \arrow[ul, ""]
    \arrow[ur, ""]
    \arrow[urr, ""]
    \arrow[urrr, ""]
    \arrow[rrr, ""]
    \arrow[lll, ""]
    & &
    &  \bigcirc \arrow[lll, ""]
    & \dots   \\[5mm]
    \dots
    & \bigcirc \arrow [urrr, ""]
    & \bigcirc \arrow[urr, ""]
    & \bigcirc \arrow[ur, ""]
    & \dots
    & \bigcirc \arrow[ul, ""]
    & \bigcirc \arrow[ull,""]
    & \bigcirc \arrow[ulll,""]
    & \dots
  \end{tikzcd}
\]
\end{document}

%%% Local Variables:
%%% mode: latex
%%% TeX-master: t
%%% End:
