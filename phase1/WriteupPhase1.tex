
\def\pathToRoot{../}\documentclass{article}

\usepackage{nag}
\usepackage[small,compact]{titlesec}
\usepackage[utf8]{inputenc}
\usepackage[T1]{fontenc}
\usepackage{lmodern}
\usepackage{color}
\usepackage{parskip}
\usepackage{needspace}
\usepackage{microtype}
\usepackage{mathtools}
\usepackage{xifthen}
\usepackage{xpatch}
\usepackage{enumitem}
\usepackage{mdwlist}
\usepackage{bussproofs}
\EnableBpAbbreviations
\usepackage{tabu}
\usepackage{amssymb}
\usepackage{amsmath}
\usepackage{amsthm}
%grober hack, der den groben hack von parskip bei den amsthm sachen korrigiert
\begingroup
    \makeatletter
       \@for\theoremstyle:=definition,remark,plain\do{%
            \expandafter\g@addto@macro\csname th@\theoremstyle\endcsname{%
                        \addtolength\thm@preskip\parskip
             }%
        }
\endgroup
\usepackage[UKenglish]{babel}
\usepackage{xparse}
\usepackage{adjustbox}
\usepackage{geometry}
\usepackage{booktabs}
\usepackage{multicol}
\usepackage{soul}
\usepackage{calc}
\usepackage{textcase}
\usepackage{stmaryrd}
\usepackage{marvosym}
\usepackage{wasysym}
\usepackage{pifont}
\newcommand{\cmark}{\ding{51}}
\newcommand{\xmark}{\ding{55}}
\usepackage{tikz}
\usetikzlibrary{trees, backgrounds, shapes, chains, decorations.text, decorations.pathreplacing, circuits.logic.IEC, patterns, matrix}
\usepackage{tikz-qtree}
\usepackage{tikzsymbols}
\usepackage{fancyvrb}
\usepackage{fancyhdr}
\usepackage{verbatim}
\usepackage[framemethod=tikz]{mdframed}
\usepackage{lastpage}
\usepackage{pgfpages}
\usepackage{csquotes}
\usepackage{longtable}
\usepackage{ragged2e}
%\usepackage{stackengine}
\usepackage{censor}
\usepackage{expl3}
\usepackage{multirow}
\usepackage{hyperref}
\usepackage{environ}


% Package for Cateogry diagrams:

\usepackage{tikz-cd}




\renewcommand{\labelenumi}{(\alph{enumi})}
\renewcommand{\labelenumii}{(\roman{enumii})}

\input{\pathToRoot headers/definitions}



\tikzset{
    normal/.style={draw, semithick},
    n/.style={style=normal, circle, inner sep=1mm, minimum size=8mm},
    l/.style={style=normal, rounded corners=1mm, inner sep=1mm, minimum size=6mm},
    e/.style={style=normal, shorten >=1mm, shorten <=1mm, ->, >=stealth},
    syntax/.style={style=normal, ellipse, minimum height=6mm, minimum width=8mm}, % nodes in syntax trees
    inner/.style={style=normal, minimum size=4mm}, % inner leaves or root in normal trees
    leaf/.style={style=normal, circle, minimum size=4mm}, % leaves in normal trees
    te/.style={style=normal}, % edges in a tree
    be/.style={style=e, dashed} % binding edge
}

\newcommand{\syntaxtree}[1]{ % DEPRECATED - use tikzsyntaxtree
    \begin{tikzpicture}[baseline=(current bounding box.north)]
        \tikzset{grow=down}
        \tikzset{every node/.style={syntax}}
        \tikzset{edge from parent/.style=
            {te,
                edge from parent path={(\tikzparentnode) -- (\tikzchildnode)}}}
        \Tree #1
    \end{tikzpicture}
}

\newenvironment{tikzsyntaxtree}[1][]{
    \begin{tikzpicture}[baseline=(current bounding box.north), #1]
    \tikzset{grow=down}
    \tikzset{every tree node/.style={syntax}}
    \tikzset{edge from parent/.style={te, edge from parent path={(\tikzparentnode) -- (\tikzchildnode)}}}
}{
    \end{tikzpicture}
}


\newcommand{\DisplayScaledProof}{\maxsizebox{\linewidth}{!}{\DisplayProof}}
\newcommand{\DisplayTopProof}{\adjustbox{valign=t}{\DisplayProof}}
\newcommand{\DisplayScaledTopProof}{\adjustbox{valign=t}{\maxsizebox{\linewidth}{!}{\DisplayProof}}}


\newcolumntype{P}[1]{>{\RaggedRight\hspace{0pt}}p{#1}}

\newenvironment{prooftable}
{
    \begin{longtable}{>{\footnotesize}p{0.33\textwidth}>{\footnotesize}p{0.33\textwidth}|>{\footnotesize}P{0.15\textwidth}}
    \normalsize Textbeweis & \normalsize Erklärungen & \normalsize Schlussregel\\\hline
    \endhead
}
{
    \end{longtable}
}


\theoremstyle{definition}
\newtheorem*{definition*}{Definition} % Definition ohne Nummer
\newtheorem*{inferenceRule*}{Schlussregel}


\geometry{a4paper,left=2cm,right=2cm,top=2cm,bottom=3cm}


\newcommand{\licenseccjuliachristian}{\def\islicenseccjuliachristian{}}
\newcommand{\suppresslicense}{\def\issuppresslicense{}}


\AtBeginDocument{
    \pagestyle{fancy}
    \renewcommand{\headrulewidth}{0pt}
    \renewcommand{\footrulewidth}{1pt}
    \fancyhead{}
    \fancyfoot[C]{\thepage~/~\pageref{LastPage}}
}


\newcommand{\pgbreakhere}{\Needspace*{4\baselineskip}}
\newcommand{\pgbreakHere}{\Needspace*{10\baselineskip}}
\newcommand{\pgbreakHERE}{\Needspace*{15\baselineskip}}

\newcommand{\raisedrule}[2][0em]{\leavevmode\leaders\hbox{\rule[#1]{1pt}{#2}}\hfill\kern0pt}

% inspired by http://tex.stackexchange.com/questions/242294/suppress-parskip-only-after-a-specific-paragraph
\makeatletter
\newlength\noparskip@parskip % used to store a backup of the parskip value
\newboolean{noparskip@triggered} % flag to indicate that noparskip was run in the current paragraph
\setboolean{noparskip@triggered}{false}
\newboolean{noparskip@active} % flag to indicate that parskip should be restored after this paragraph
\setboolean{noparskip@active}{false}
\let\noparskip@par\par % store a backup of the \par command
\@setpar{% redefine \par with the means of ltpar.dtx to stay compatible to enumerate and itemize
    \ifhmode% since we're counting occurrences of \par, \par\par would be a problem, so check that we are actually ending a paragraph
        \ifthenelse{\boolean{noparskip@active}}{%
            \setlength\parskip\noparskip@parskip% restore parskip
            \setboolean{noparskip@active}{false}% remember not the restore parskip again
        }{}%
        \ifthenelse{\boolean{noparskip@triggered}}{%
            \ifthenelse{\boolean{noparskip@active}}{}{
                % we are triggering noparskip and not currently in a noparskip already
                \setlength\noparskip@parskip\parskip % copy the current parskip into the backup variable
            }%
            \setboolean{noparskip@triggered}{false}% paragraph is ending, so noparskip is no longer triggered
            \setlength\parskip{0pt}% no parskip when the next paragraph begins
            \setboolean{noparskip@active}{true}% parskip must be restored by the next par
        }{}%
    \fi%
    \noparskip@par% run the original par command
}
\def\noparskip@backout{%
    \ifthenelse{\boolean{noparskip@active}}{%
        % a list is beginning and parskip is currently set to zero, wich would mess up the list
        \setlength\parskip{\noparskip@parskip}% restore parskip before the list begins
        \setboolean{noparskip@active}{false}%
    }{}%
    \setboolean{noparskip@triggered}{false}% there's no sense in keeping noparskip triggered throughout a list
}
\xpretocmd\begin{%
    \ifstrequal{#1}{enumerate}{\noparskip@backout}{}%
    \ifstrequal{#1}{itemize}{\noparskip@backout}{}%
    \ifstrequal{#1}{list}{\noparskip@backout}{}%
    \ifstrequal{#1}{proof}{\noparskip@backout}{}%
}{}{}
\def\noparskip{%
    \leavevmode% ensure that we are within a paragraph
    \setboolean{noparskip@triggered}{true}% trigger noparskip
}
\makeatother

\newcommand{\noparskipworkaround}{} % DEPRECATED and no longer needed


\newcommand{\head}[1]{
    {
        \setlength{\parskip}{0pt}
        \hrule height 1pt
        \vspace{.2cm}
        Saarland University \hfill Category Theory Seminar 2017\par
        Programming Systems Lab \hfill \small\url{https://courses.ps.uni-saarland.de/ct_ss17/}\par
        \tiny\raisedrule[0mm]{1pt}
        \vspace{2ex}
        \begin{center}
            \Large
            \textbf{#1}\par
            \raisedrule[2mm]{1pt}
        \end{center}
        \vspace{3ex}
    }
}

\newenvironment{leftframedparagraph}{\begin{mdframed}[hidealllines = true, leftline = true, innerleftmargin = 2ex, innerrightmargin = 0pt,
innertopmargin = 0pt, innerbottommargin = 2pt, skipabove=2ex, skipbelow=1ex, outerlinewidth = 0ex, innerlinewidth = 0.5ex]}{\end{mdframed}}
\newenvironment{leftframed}{\begin{mdframed}[hidealllines = true, leftline = true, innerleftmargin = 2ex, innerrightmargin = 0pt,
innertopmargin = 0pt, innerbottommargin = 0pt, skipabove=2ex, skipbelow=1ex, outerlinewidth = 0ex, innerlinewidth = 0.5ex]}{\end{mdframed}}

%%% Local Variables:
%%% mode: latex
%%% TeX-master: t
%%% End:


\newcommand{\uebunghead}[3][Exercise sheet:]{\def\sheetid{#2}\head{#1 #2\ifthenelse{\isundefined{\issolution}}{}{ \ifthenelse{\isundefined{\ismarking}}{(Possible solutions)}{(Marking)}} \\ #3}}

\licenseccjuliachristian


\newcommand{\amountofpoints}[1]{\ifstrequal{#1}{1}{1~Punkt}{#1~Punkte}}


% marking implies solution
\ifthenelse{\isundefined{\ismarking}}{}{\def\issolution{}}


%%%Environments
\newcounter{ExamExerciseCounter} % will only be used in exams, but must be defined here so ExerciseCounter can be reset when ExamExericise counts
\setcounter{ExamExerciseCounter}{0}
\newcounter{ExerciseCounter}[ExamExerciseCounter]
\setcounter{ExerciseCounter}{0}

\newcommand{\ExerciseNumber}{\sheetid.\arabic{ExerciseCounter}}
\renewcommand{\theExerciseCounter}{\ExerciseNumber}

\newcommand{\ExercisePointHook}[1]{}

%Aufgaben-Umgebung
\NewDocumentEnvironment{exercise}{od<>}{
    \refstepcounter{ExerciseCounter}
    \pgbreakhere
    \vspace{1ex}\textbf{Exercise\ \ExerciseNumber}%
    \IfNoValueF{#1}{ \emph{(#1)}}%
    \IfNoValueF{#2}{\hfill(\amountofpoints{#2})}%
    \IfNoValueF{#2}{\ExercisePointHook{#2}}%
    \noparskip\par\nopagebreak
}{
    \par
    \vspace{2ex}
}



%Loesungs-Umgebung
\newenvironment{answer}
{
    \ifthenelse{\isundefined{\issolution}}
    {
        \comment
    }{
        \vspace{1ex}\textsl{Lösungsvorschlag \ExerciseNumber}\noparskip\par\nopagebreak
    }
}{
    \ifthenelse{\isundefined{\issolution}}
    {
    }{
        \vspace{1ex}
        \hspace*{\fill}
    }
}

\newenvironment{marking}
{%
    \ifthenelse{\isundefined{\ismarking}}%
    {%
        \comment%
    }{%
        \color{red}
    }%
}{%
    \ifthenelse{\isundefined{\ismarking}}%
    {%
    }{%
    }%
}

\newenvironment{example}{\begin{leftframedparagraph}\paragraph{Example:}}{\end{leftframedparagraph}}
\newenvironment{hint}{\paragraph{Hint:}}{}
\newenvironment{caution}{\paragraph{Caution:}}{}
\newenvironment{definition}[1]{\begin{leftframedparagraph}\paragraph{Definition (#1):}}{\end{leftframedparagraph}}


\usepackage{charter}
\fontfamily{bch}\selectfont

\begin{document}
\title{What is a Category?}
\author{Sarah Mameche, Leonhard Staut, Andreas Meyer}
\maketitle

\section {Definition of Categories}
\begin{definition}{Category}
	A category $\mathscr{C}$ has the following components:
	\begin{itemize}
		\item a collection of objects $ob(\mathscr{C})$
		\item $\forall \; A, B \in ob(\mathscr{C})$, a collection of maps (arrows, morphisms), denoted by‚ $\mathscr{C}(A, B)$
		\item $\forall \; A, B, C \in ob(\mathscr{C})$, a composition function
		\[\mathscr{C}(B, C) \times \mathscr{C}(A, B) \rightarrow  \mathscr{C}(A, C)
		\]		\[ (g,f) \mapsto (g \of f)
		\]
		\item $\forall A \in ob(\mathscr{C})$, an element $1_A \in \mathscr{C}(A, A)$, called the identity on $A$ 
 	\end{itemize}
 The following axioms have to be satisfied:
 \begin{itemize}
 	\item identity laws: $\forall f \in \mathscr{C}(A, B). \; f \of 1_A = f = 1_B \of f$ 
 	\item associativity: $\forall f \in \mathscr{C}(A, B), \; g \in \mathscr{C}(B, C), \; h \in \mathscr{C}(C, D). \; (h \of g) \of f = h \of (g \of f) $.
 \end{itemize}
\end{definition}

\section {Diagrams}
It is often possible and useful to draw diagrams for a category or parts thereof.
For example consider the following diagram.
\[
  \begin{tikzcd}
    A \arrow{r}{f} \arrow{d}{g} & B \arrow{d}{h} \\
    C \arrow{r}{i}              & D 
  \end{tikzcd}
\]
This represents the category with objects $\{A, B, C, D\}$ and
morphisms $\{f, g, h, i, 1_A, 1_B, 1_C, 1_D \}$.
Since every object is required to have an identity arrow by definition,
we usually leave them implicit and do not include them in diagrams.\\
We say a diagram \emph{commutes} if any two paths between two objects
obtained by composing arrows are the same.
The example diagram above commutes if $ h \circ f = i \circ g $, meaning there is
exactly one way to reach $D$ from $A$.

\section {Epics and Monics}

Monomorphisms and epimorphisms are two special kinds of morphisms.

\begin{definition}{Monomorphism}
  Let $\mathscr{C}$ be a category. A map $X \overset{f}{\to} Y$ is called a monomorphism if for all objects $Z$ and maps
  $
  \begin{tikzcd}
    Z \arrow[r,shift left, "g"] \arrow[r,shift right,swap,"g'"] & X
  \end{tikzcd}
  $
  \[
    f \circ g = f \circ g' \Rightarrow g = g'
  \]
  If $f$ is a monomorphism, we say $f$ is monic.
\end{definition}

\begin{definition}{Epimorphism}
  Let $\mathscr{C}$ be a category. A map $X \overset{f}{\to} Y$ is called an epimorphism if for all objects $Z$ and maps
  $
  \begin{tikzcd}
    Y \arrow[r,shift left, "g"] \arrow[r,shift right,swap,"g'"] & Z
  \end{tikzcd}
  $
  \[
    g' \circ f = g' \circ f \Rightarrow g = g'
  \]
  If $f$ is an epimorphism, we say $f$ is epic.
\end{definition}

%% TODO: We should probably add a reference to the definition of categories, so that we can link it in the text below
%% TODO: Distribute examples better, in particular we should introduce the dual category earlier, because
%% Monics and epics are dual.
%% TODO: Possibly add a proof why epic maps in Set are surjective functions
%% TODO: Possibly add a proof that there are non-surjective epic maps in Ring

In other words, a monomorphism is cancellable on the left, and an epimorphism is cancellable on the right of compositions.
By the identity laws of categories, for any object $A$ the identity map $1_A$ is always both monic and epic.\\
In the category \textbf{Set}, monic maps are exactly the injective functions, and epic maps are exactly the surjective functions.
In ordinary set theory these properties of functions are formulated in terms of elements of a set.
Epic and monic arrows enable us to talk about these properties without referring to elements at all.
In the categories \textbf{Grp}, \textbf{Mon}, \textbf{Ring}, and \textbf{Vect}, monic maps are exactly the injective homomorphisms.
However, in \textbf{Ring} there are maps that are epic, but not surjective.

\section {Isomorphisms}
\begin{definition}{Isomorphism}
  Let $\mathscr{C}$ be a category and $A, B \in ob(\mathscr{C})$.
  \\ A map $f \from A \to B$ in $\mathscr{C}$ is an isomorphism if there is a $g \from B \to A$ s.t. $g \circ f = 1_A$ and $f \circ g = 1_B$.
  \\ $A$ and $B$ are isomorphic in $\mathscr{C}$ ($A \cong B$) if there is an isomorphism between them.
\end{definition}

In the category \textbf{Set} isomorphisms are exactly the invertible functions. It can be shown that invertible functions between two sets are
exactly the bijective functions, i.e. functions that are both injective and surjective.
Therefore, isomorphisms in the category \textbf{Set} are exactly those maps that are both monic and epic.\\
In the category \textbf{Grp} isomorphisms are exactly the bijective group homomorphisms.
since those are the group homomorphisms that have an inverse homomorphism.\\
Similarly, in the category \textbf{Ring} isomorphisms are exactly the bijective ring homomorphisms.\\[1mm]
We now show for a general category $\mathscr{C}$ that every isomorphism is both monic and epic.
\begin{proof}
  Consider the diagram:
  \[
    \begin{tikzcd}
      A \arrow[r,shift left, "y"] \arrow[r,shift right,swap,"x"] &
      B \arrow[dr, shift left, "1"]  \arrow{r}{m}  &
      C \arrow{d}{e} \arrow[r,shift left, "i"] \arrow[r,shift right,swap,"j"] & D\\
      &  & B \arrow[ul, shift left, "1"]&
    \end{tikzcd}
  \]
  $m \from B \to C$ is an isomorphism with inverse $e \from C \to B$.
  Then:\\
  \begin{minipage}{.5\linewidth}
    \vspace{4mm}
    \centering $m$ is monic:
    \[
      \begin{aligned}
        m \of x &= m \of y\\
        e \of (m \of x) &= e \of (m \of y)\\
        (e \of m) \of x &= (e \of m) \of y\\
        1_B \of x &= 1_B \of y\\
        x &= y
      \end{aligned}
    \]
  \end{minipage}%
  \begin{minipage}{.5\linewidth}
    \vspace{4mm}
    \centering $m$ is epic:
    \[
      \begin{aligned}
        i \of m &= j \of m \\
        (i \of m) \of e &= (j \of m) \of e\\
        i \of (m \of e) &= j \of (m \of e)\\
        i \of 1_C &= j \of 1_C\\
        i &= j
      \end{aligned}
    \]
  \end{minipage}%
\end{proof}
Note that the converse of the statement holds for the category \textbf{Set}, but not in general.
If a map is both monic and epic it is not necessarily an isomorphism.
For a map $f$ to be an isomorphism there needs to exist an inverse map $f^{-1}$.
However monic and epic maps do not necessarily have inverses.\\[1mm]
If a map $f$ is an isomorphism, then its inverse is unique. Thus we speak of
\emph{the} inverse of $f$.

\begin{proof}
  Let $f \from A \to B$ be a map.
  Assume there exist two inverses $g \from B \to A$, such that $gf = 1_A$, $fg = 1_B$ and \mbox{$h \from B \to A$}, such that $hf = 1_A$, $fh = 1_B$.
  We know from the identity axiom: $1_A \of g = g$ and $h \of 1_B = h$. Therefore we get:
 \[  1_A \of g = g \textiff (h \of f) \of g = g \textiff h \of (f \of g) = g \textiff h \of 1_B = g \textiff h=g. \]
\end{proof}

\section {Examples}
\begin {itemize}

 \item The category of sets, denoted as \textbf{Set}, is the category whose objects $ob$(\textbf{Set}) are sets.
   \\ The arrows in \textbf{Set} are the functions between two $A,B\ \in ob(\textbf{Set}).$
   \\ The identity- function is defined as $\forall A: set, \ id_A:\ A \to A,\ \forall x \in A \ f \ x = x$
   \\ The composition $\circ$ is the composition of functions and this is associative. Proof:
   \\ Assume sets $A, B, C, and D$ and functions $f \from A \to B$, $g \from B \to C$, $h \from C \to D$ and $x \in A$.
   \\ $((h\ \circ \ g)\ \circ\ f)\ x = ((h \ \circ g \ )(f \ x) =h(g(f\  x)) $
   \\ $(h\ \circ \ (g \ \circ f))\ x = h ((g\ \circ \ f) \ x) = h(g(f \ x))$
   \\ Thus $(h\ \circ \ g)\ \circ \ f = \ h \ \circ \ (g \ \circ \ f)$ holds. $\Box$
 \item The category of relations, denoted as \textbf{Rel}, is the category whose objects are the sets
   \\The arrows are all binary relations between two $A,B \ \in ob(\textbf{Rel})$
   \\The identity arrow  is the identity function $\forall A: set, \ id_A:\ A \to A,\ \forall x \in A \ f \ x = x$
   \\ The composition $R \circ S$, $R  \in \textbf{Rel}(A,B), S \in \textbf{Rel}(B,C), A,B,C \in ob(\textbf{Rel})$ is defined as $ (x,y) \in  R \circ S \leftrightarrow \exists  z.(x,z) \in S \land (z,y) \in R$
   
  \item Categories similar to \textbf{Set} can be constructed for sets which have some additional structure and structure-preserving mappings between them. For example, there is a category \textbf{Poset} with partially ordered sets as objects. It is defined similarly to \textbf{Set}, with the difference that the functions are monotone. 
  
  \item A single poset $(P, \leq)$ also gives rise to a category $\mathscr{P}$ if we take elements of $P$ to be the objects. There is a unique arrow in $\mathscr{P}(A, B)$ iff $A \leq B$. The reflexivity requirement of $\leq$ ensures that identity arrows exist for all objects. Also, the category has composition because the order is transitive.
  
  \item \textbf{Vect$_k$} is the category of vector spaces over a field $k$, with linear transformations as arrows.
  
  \item There is a category \textbf{Grp} whose objects are groups; maps are given by group homomorphisms between two groups $A,B \in ob(\textbf{Grp}).$ 
  
  \item A single group $(G, \cdot)$ can also be seen as a category $\mathcal{G}$. There is a unique object $G$ which represents the group itself, and arrows in $\mathcal{G}(G, G)$  correspond to group elements. \\
    The identity arrow is the unit element $1$ of the group.\\
    Composition operates on two group elements and has to be associative, so it is given by the group action.\\
    A group furthermore requires all elements to be invertible, thus we need each arrow $\mathcal{G}(G, G)$ to be an isomorphism.\\
    Dropping the last requirement, we get the category of a single monoid.
    %% TODO: explain how this is interesting because objects don't have to be set-like

  
  \item Categories need not represent mathematical structures. They can be arbitrarily constructed by giving objects, arrows and arrow composition in a way that satisfies the axioms in the definition. For instance, the category \textbf{1} includes a single object and its identity map, without further specification what the object is. \\
    A category without non-trivial arrows is called discrete; removing all arrows except identities makes an arbitrary category discrete.
 \end {itemize}

 \section {Dual categories}

 $\mathscr{C}^{op}$ denotes the dual category for any category $\mathscr{C}$. One obtains $\mathscr{C}^{op}$ by reversing the arrows in $\mathscr{C}$. For every sentence $\Sigma$ in the language of category theory, the reversed sentence $\Sigma^*$ exists, and any proof for any theorem yields a proof for the dual theorem by the duality principle.

\section {Terminal and Initial Objects}
\begin {definition}{Initial and terminal Objects}
  In any category C,
  \ an object $0$ is called initial iff for any object $A \in C$, there is a unique morphism $0 \to A$.
  \ an object $1$ is called terminal iff for any object $A \in C$, there is a unique morphism $A \to 1$. A terminal object in C is initial in $C^{op}$.

\end {definition}

\subsection {Proposition:}
   Initial and terminal objects are unique up to isomorphism.

\subsubsection {Proof:}
   Assume $0, 0'$ are both inital objects in some category C and show that $f: 0 \to 0', g: 0' \to 0$ form a unique isomorphism $f \circ g$ between 0,0'. One can draw the following diagram:
  
\begin{tikzcd}
      & 0' \arrow{dr}{g} \arrow{rr}{id_{0'}}    &&  0'\\
      0 \arrow{ur}{f} \arrow{rr}{id_0} && 0 \arrow{ur}{f}
    \end{tikzcd}
 

  Since $0$ is initial, we know that $f$ is unique, from the same argument follows uniqueness of $g = f^{-1}$. Therefore $f \circ g$ and $g \circ f$ is unique.    
    \\ The same holds for terminal objects by duality. $\Box$

        Categories in which the terminal is identical to the initial object are called pointed category. Such objects zero objects.

    \subsubsection {Example:}
    How to show that $\emptyset$ is initial in Set and the one- element set $\{x\}$ terminal?

    \begin {itemize}
    \item There is only the binary union function from $\emptyset$ to any other set, since there are no arguments in the domain to use.
    \item Assume some f with
      \\ $\forall A:$ set with $A \neq \emptyset$, $ f: A \to \{x\}$
      f is obviously a constant function, since $\forall y \in A,\ f \ y = x$ holds, and therefore unique.
		\end{itemize}
\end{document}

%%% Local Variables:
%%% mode: latex
%%% TeX-master: t
%%% End:
