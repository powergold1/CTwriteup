
\def\pathToRoot{../}\input{\pathToRoot headers/uebungHeader}


\begin{document}
\title{What is a Category?}
\author{sarah, leonhard, Andreas Meyer}
\maketitle

\section {Category}

TODO

\section {Epis and Monis}

TODO

\section {Isomorphism}

\section {Examples}
\begin {itemize}

 \item The category of sets, denoted as \textbf{Set} is the category whose objects ob(\textbf{Set}) are sets.
   \\ The arrows in \textbf{Set} are the functions between two $A,B\ \in ob(\textbf{Set}).$
   \\ The identity- function is defined as $\forall A: set, \ id_A:\ A \to A,\ \forall x \in A \ f \ x = x$
   \\ The composition $\circ$ is the composition of functions and this is associative. Proof:
   \\ Assume sets A,B,C,D, functions $f:\ A \to B$, $g:\ B \to C$, $h:\ C \to D$ and x $\in$ A.
   \\ $((h\ \circ \ g)\ \circ\ f)\ x = ((h \ \circ g \ )(f \ x) =h(g(f\  x)) $
   \\ $(h\ \circ \ (g \ \circ f))\ x = h ((g\ \circ \ f) \ x) = h(g(f \ x))$
   \\ Thus $(h\ \circ \ g)\ \circ \ f = \ h \ \circ \ (g \ \circ \ f)$ holds. \Box \\

 \item The category of relations, denoted as \textbf{Rel} is the category whose objects are the sets
   \\The arrows are all binary relations between two $A,B \ \in ob(\textbf{Rel})$
   \\The identity arrow  is the identity function $\forall A: set, \ id_A:\ A \to A,\ \forall x \in A \ f \ x = x$
   \\ The composition $R \circ S$, $R  \in \textbf{Rel}(A,B), S \in \textbf{Rel}(B,C), A,B,C \in ob(\textbf{Rel})$ is defined as $ (x,y) \in  R \circ S \leftrightarrow \exists  z.(x,z) \in S \land (z,y) \in R$
  \item TODO Grp% {Category of Groups}
  \item TODO Vec% {Category of Vectorspaces}
  \item TODO Pos% {Category of Partial Ordered Sets}
  \item TODO Discrete, 1, 2% {Discrete Category, Category 1,2}
 \end {itemize}

 \section {Dual categories}

 $C^{op}$ denotes the dual category for any category C. One obtains $C^{op}$ by reversing the arrows in C. For every sentence $\Sigma$ in the language of category theory, the reversed sentence $\Sigma^*$ exists, therefore any proof for any theorem yields for the dual theorem by the duality principle.  

\section {Terminal and Initial Objects}
\begin {definition}{Initial and terminal Objects}
  In any category C,
  \ an object 0 is called initial iff for any object $A \in C$, there is an unique morphism $0 \to A$.
  \ an object 1 is called terminal iff for any object $A \in C$, there is an unique morphism $A \to 1$. A terminal object in C is initial in $C^{op}$

\end {definition}

\subsection {Proposition:}
  \\ Initial and terminal objects are unique up to isomorphism.

\subsubsection {Proof:}
  \\ Assume 0,0' are both inital objects in some category C and show that $f: 0 \to 0', g: 0' \to 0$ form an unique isomorphism $f \circ g$ between 0,0'. One can draw the following diagram:
  
\begin{tikzcd}
      & 0' \arrow{dr}{g} \arrow{rr}{id_{0'}}    &&  0'\\
      0 \arrow{ur}{f} \arrow{rr}{id_0} && 0 \arrow{ur}{f}
    \end{tikzcd}
 

  Since 0 is initial, we know that f is unique, from the same argument follows uniqueness of $g = f^{-1}$. Therefore $f \circ g$ and $g \circ f$ is unique.    
    \\ The same holds for terminal objects by duality. $\Box$

    \\    Categories, in which the terminal is identical to the initial object, are called pointed category. Such objects zero objects.

    \subsubsection {Example:}
    How to show that $\emptyset$ is initial in Set and the one- element set $\{x\}$ terminal?

    \begin {itemize}
    \item There is only the binary union function from $\emptyset$ to any other set, since there are no arguments in the domain to use.
    \item Assume some f with
      \\ $\forall A:$ set with $A \neq \emptyset$, $ f: \A \to \{x\}$
      f is obviously a constant function, since $\forall y \in A,\ f \ y = x$ holds, and therefore unique.

\end{document}

%%% Local Variables:
%%% mode: latex
%%% TeX-master: t
%%% End:
